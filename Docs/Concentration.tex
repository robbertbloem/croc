\documentclass[11pt, fleqn]{report}
\usepackage{graphicx}
\usepackage{amssymb}
\usepackage{amsfonts}
\usepackage{amsbsy}
\usepackage{amsmath}

\newcommand{\pc}{[P]}
\newcommand{\lc}{[L]}
\newcommand{\plc}{[PL]}
\newcommand{\kd}{k_d}
\newcommand{\ka}{k_a}
\newcommand{\pcz}{\lbrack P_T\rbrack}
\newcommand{\lcz}{\lbrack L_T\rbrack}
\newcommand{\xr}{X_r}

\begin{document}

For the reaction
\begin{eqnarray}
A+B \stackrel{k_a}{\rightleftharpoons} AB
\end{eqnarray}
we can calculate $k_d$ as follows:
\begin{eqnarray}
\frac{[A][B]}{[AB]} = \frac{1}{k_a} = k_d \label{eq:kd}
\end{eqnarray}
Using
\begin{eqnarray}
\lbrack A_0\rbrack = [A] + [AB] \\
\lbrack B_0\rbrack = [B] + [AB] 
\end{eqnarray}
we rewrite eqn \ref{eq:kd} so that it can be solved with the square (ABC) formula:
\begin{eqnarray}
\frac{([A_0]-[AB])([B_0]-[AB])}{[AB]} = k_d \\
\lbrack A_0 \rbrack[B_0]-[AB]([A_0]+[B_0])+[AB]^2= k_d [AB] \\
\lbrack AB \rbrack^2 - [AB]([A_0]+[B_0]+k_d) + [A_0][B_0] = 0
\end{eqnarray}
Percentage bound is:
\begin{eqnarray}
\frac{[AB]}{[A_0]}\cdot 100\%
\end{eqnarray}


\section*{Determine $k_a$}

According to paper Indyk (1998).

Protein, $M_T$ is now $P_0$. Ligand, $L_T$ is now $L_0$. The binding constant $k_b$ is now $k_a$. 

Reaction
\begin{eqnarray}
P+L \stackrel{k_a}{\rightleftharpoons} PL
\end{eqnarray}
with binding constant $k_a$
\begin{eqnarray}
\frac{\plc}{\pc \lc} = \frac{1}{k_d} = k_a \label{eq:ka}
\end{eqnarray}

\begin{eqnarray}
\pcz = \pc + \plc \\
\lcz = \lc + \plc \\
\plc = \pc \lc \ka \\
\pcz = \pc + \pc \ka \lc = \pc(1+\ka\lc)\\
\lcz = \lc + \lc \ka \pc \\
\frac{\pcz}{1+\ka\lc} = \pc = \frac{\lcz-\lc}{\ka\lc} \\
\ka\pcz\lc = (\lcz-\lc)(1+\ka\lc)
\end{eqnarray}
This yields
\begin{eqnarray}
\ka\pcz\lc - \lcz - \lcz\ka\lc + \lc + \ka\lc^2 = 0 \\
\ka\lc^2 + \ka\pcz\lc - \lcz\ka\lc + \lc - \lcz = 0 
\end{eqnarray}

\subsection*{Following the paper}
Am I stupid? In the paper, from eq 9 to 10 doesn't work. Anyway, to continue with eq. 10.
\begin{eqnarray}
\lc^2 + \lc ( \pcz - \lcz + 1/\ka) - \lcz = 0 
\end{eqnarray}
We fill this in the quadratic equation:
\begin{eqnarray}
a=1\\
b=\pcz - \lcz + 1/\ka\\
c=-\lcz\\
\end{eqnarray}
Huh?
\begin{eqnarray}
\lc=\frac{-(\lcz-\pcz-1/\ka)\pm\sqrt{(\pcz-\lcz-1/\ka)^2+4\lcz)}}{2}
\end{eqnarray}
We divide by $\pcz$ and define $\xr=\lcz/\pcz$ and $r=1/(\ka\pcz)$
\begin{eqnarray}
\frac{\lc}{\pcz} = \frac{\frac{\lcz}{\pcz}-\frac{\pcz}{\pcz}-\frac{1}{\ka\pcz} \pm \sqrt{(\frac{\pcz}{\pcz}-\frac{\lcz}{\pcz}+\frac{1}{\ka\pcz})^2+4\frac{\lcz}{\pcz}}}{2}\\
\frac{\lc}{\pcz} = \frac{\xr - 1 - r \pm \sqrt{(1 - \xr + r)^2 + 4\xr}}{2}
\end{eqnarray}
This again deviates from the paper, where the signs of $X_r$ are the other way around.

The calorimeter measures $d q / d \lcz$, where $q$ is heat.

Since
\begin{eqnarray}
\frac{d\lc}{d\lcz}=\frac{d\lc}{d\xr}\frac{d\xr}{d\lcz}\\
\frac{d\xr}{d\lcz}=\frac{1}{\pcz}
\end{eqnarray}
the derivative is
\begin{eqnarray}
\frac{d\lc}{d\xr} = \left( \frac{1}{2} + \frac{\xr+r-1}{\sqrt{(\xr+r+1)^2-4\xr}} \right) \pcz \\
\frac{d\lc}{d\lcz} = \left( \frac{1}{2} + \frac{\xr+r-1}{\sqrt{(\xr+r+1)^2-4\xr}} \right) 
\end{eqnarray}


\subsection*{My own derivation}
\begin{eqnarray}
\lc^2 + \lc ( \pcz - \lcz + 1/\ka) - \lcz/\ka = 0 
\end{eqnarray}
We fill this in the quadratic equation:
\begin{eqnarray}
a=1\\
b=\pcz - \lcz + 1/\ka\\
c=-\lcz/\ka\\
\lc=\frac{-\left( \pcz-\lcz+1/\ka \right)\pm \sqrt{(\pcz-\lcz+1/\ka)^2+4\lcz/\ka)}}{2}
\end{eqnarray}
We divide by $\pcz$ and define $\xr=\lcz/\pcz$ and $r=1/(\ka\pcz)$
\begin{eqnarray}
\frac{\lc}{\pcz} = \frac{ -\frac{\pcz}{\pcz} + \frac{\lcz}{\pcz} - \frac{1}{\ka\pcz} \pm  \sqrt{\left( \frac{\pcz}{\pcz}-\frac{\lcz}{\pcz}+\frac{1}{\ka\pcz} \right)^2+4\frac{\lcz}{\ka\pcz}}}{2}\\
\frac{\lc}{\pcz} = \frac{-1 + \xr -r \pm \sqrt{(1 - \xr + r)^2+4\xr/\ka}}{2}
\end{eqnarray}
the derivative is
\begin{eqnarray}
\frac{d\lc}{d\xr} = \left( \frac{1}{2} + \frac{\ka(\xr-r-1)+2}{ 2\ka\sqrt{(-\xr+r+1)^2 + 4\xr/\ka} } \right) \pcz\\
\frac{d\lc}{d\lcz} = \left( \frac{1}{2} + \frac{\ka(\xr-r-1)+2}{ 2\ka\sqrt{(-\xr+r+1)^2 + 4\xr/\ka} } \right)
\end{eqnarray}

For one reason or the other, they took $X_r=-X_r$. 


\subsection*{Calculating the heat}

The expression for the total heat that has been released is
\begin{eqnarray}
q=\plc \Delta H V
\end{eqnarray}
The machine measures $d q /d \lcz$:
\begin{eqnarray}
\frac{d q}{d\lcz} = \Delta H V \frac{d\plc}{d\lcz} \\
\frac{d\plc}{d\lcz} = \frac{d\lcz}{d\lcz} - \frac{d\lc}{d\lcz} 
\end{eqnarray}
Which is what we calculated above.

We should take the positive part of the quadratic function.

Plugging things into each other:
\begin{eqnarray}
\frac{d q}{d\lcz} = \Delta H V \left( \frac{1}{2} + \frac{\ka(-\xr-r-1)+2}{ 2\ka\sqrt{(\xr+r+1)^2 - 4\xr/\ka} } \right)
\end{eqnarray}


\end{document}



























